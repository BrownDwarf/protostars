\documentclass[twocolumn]{emulateapj}% change onecolumn to iop for fancy, iop 

\let\pwiflocal=\iffalse \let\pwifjournal=\iffalse
\input{mgs_setup}

\newcommand{\iancze}{{\sc C15}}

\providecommand{\eprint}[1]{\href{http://arxiv.org/abs/#1}{#1}}
\providecommand{\adsurl}[1]{\href{#1}{ADS}}
\newcommand{\project}[1]{\textsl{#1}}

\slugcomment{In preparation}
\shorttitle{Class 0 near-IR spectroscopy}
\shortauthors{TBD}

\bibliographystyle{yahapj}

\begin{document}
 
\title{Near infrared spectrum of a Class 0 protostar photosphere}

\author{T.B. Determined,\altaffilmark{1}}


\altaffiltext{1}{NASA Ames}

\begin{abstract}
We measure the stellar photosphere of a Class 0 protostar.
\end{abstract}

\keywords{stars: fundamental parameters --- stars: individual (S68N) ---  stars: low-mass -- stars: statistics}

\maketitle

\section{Introduction}\label{sec:intro}
The method is from \citep{czekala16}.  It is similar to \citep{2017ApJ...836..200G}.

\subsection{Protostars}
\subsection{Previous studies of protostar photospheres}
\subsection{Previous studies of S68N}


\section{Observations}
The observations of S68N were carried out with NIRSPEC on the Keck telescope over 5 epochs from 2003-2014.  The \texttt{NIRSPEC-7} filter provided 1.84$-$2.63 $\mu$m wavelength range, with native spectral resolution of $R=1250$ from a 0\farcs76 (4 pixels) $\times$ 42\farcs0 slit.  
The individual epochs were coadded to form a single composite spectrum of signal to noise ratio $S/N\sim30$.  

\section{Method}

Our approach involves Bayesian forward modeling of the observed spectrum under a range of different assumptions about the prostellar system.  We discuss the limitations of our assumptions in the discussion section.

\subsection{Composite spectrum forward model}

We assume that stellar photosphere and warm circumstellar disk emission dominate the radiation in the near-IR band.  We constructed a forward model for the observed Keck spectrum using the spectral inference framework \texttt{Starfish} from \citet{czekala16} with support for mixture model modifications from \citet{2017ApJ...836..200G}.  The mixture model for S68N's Keck spectrum is:

\begin{eqnarray} \label{eqn:mix_M}
	S_{\mathrm{mix}} = \Omega_{\mathrm{\star}} B(\teffa)  + \Omega_{\mathrm{d}} B(\teffb)
\end{eqnarray}
where $B(T)$ is the spectral radiance as a function of temperature.  The ratio of solid angles relates the relative size of the emission regions:
\begin{eqnarray} \label{eqn:fill_factor}
	\frac{\Omega_d}{\Omega_\star} = \frac{r_d^2 d^2}{r_\star^2 d^2} = \frac{r_d^2}{r_\star^2}
\end{eqnarray}

approximating the circumstellar disk emission as a face-on circle.  More accurate estimates for the disk emission and geometries could be obtained from rigorous radiative transfer methods \citep[\emph{e.g.}]{2017arXiv170305765R}.  We model the protostellar spectral radiance $B(T_\star)$ with \PHOENIX\ synthetic model spectra possessing an unknown-but-low surface gravity $\logg \in [2.0,4.0]$ in cm$/$s$^2$ and an effective temperature $\teffa \in [2700,3600]$ K.  The circumstellar disk radiance is assumed to be black body, with a temperature range $\teffb \in [1000,1700]$ K capped by the temperature at which typical dust sublimates.

The radiated spectrum passes through an unknown amount of extinction and scattering, altering its spectral shape in an uncertain---and assumed irretrievable---way.  So instead of characterizing the spectral slope by a reddening law and extinction, $A_V$, we instead assume that the overall spectral shape is approximated by a third order polynomial \footnote{The polynomial is parameterized as a linear combination of Chebyshev polynomials truncated to the first 3 non-constant terms.}.

We assume there is zero emission attributable to mass accretion onto the star or disk.  This assumption almost certainly fails for Class 0 protostars, which are still actively accreting a large fraction of their mass.  However, the dearth of any conspicuous emission lines suggests that perhaps the S68N is going through a quiescent phase in a variable accretion history.  In any case, the limited spectral information available does not permit a more detailed treatment of accretion or disk gas emission as is possible with higher spectral resolution \citep[\emph{e.g.}]{2016ApJ...826..179L}.  



\subsection{Experiments}
 We tried several experiments:
%%%%%%%%%%%%%%%%%%%%%%%%%%%%%%%%%%%%%%%%
% TABLE - Experiment log
%%%%%%%%%%%%%%%%%%%%%%%%%%%%%%%%%%%%%%%%
\begin{deluxetable*}{ccp{9cm}p{5cm}}

\tabcolsep=0.11cm
%\rotate
\tablecaption{Exeriment log \label{tbl_history}}
\tablewidth{0pt}
\tablehead{
\colhead{Exp.} &
\colhead{Run.} &
\colhead{Desc.} &
\colhead{Outcome}
\\
\colhead{\#} &
\colhead{\#} &
\colhead{-} &
\colhead{-}
}
\startdata
 1 & 1 & Original, flawed \texttt{star\_veil.py} code modeled disk radiance as a constant value with no physical interpretation & Exposed need for more physical inputs \\
2 & 1 & Run with \texttt{star\_BB.py} code with erroneous $\Omega$ prescription, weak priors, poor first guess & Exposed need for better priors and starting guess \\
2 & 2 & Run with \texttt{star\_BB.py} code with erroneous $\Omega$ prescription, strong priors ($T_{d}$ fixed to $\sim1100$ K), good first guess & Exposed incorrect $\Omega$ treatment \\
3 & 1 & Run with \texttt{star\_BB.py} code with correct absolute flux level, strong priors ($T_{d}$ fixed to $\sim1100$ K). & Reasonable results, weak constraints on $\logg$. \\
3 & 2 & \emph{In progress:} Run with \texttt{star\_BB.py} code with correct absolute flux level, \emph{weak} priors ($T_{d}$ variable $\in [1000,1700]$ K). & -- \\
4 & 1 & Reduced spectral range to enhance $\logg$ sensitivity & Good convergence, wide range \\
\enddata
\end{deluxetable*}

\section{Results}

\subsection{Experiment 3, Run 1}


\begin{figure}
 \centering
 \includegraphics[width=0.45\textwidth]{../plots/logg_omega_phys.pdf}
 \caption{Posterior probability distribution function from Experiment 3, Run1, marginalized over all stellar and nuisance parameters except $\logg$ and $\Omega_d/\Omega_\star$.  The surface gravity correlates with the solid angle ratio since both of these factors affect the CO line depths.  Lines of differing stellar mass and disk radii are shown for a range of stellar radii.}
 \label{fig:posterior_loggOmega}
\end{figure}


~\clearpage

\begin{figure*}[b]
 \centering
 \includegraphics[width=0.85\textwidth]{../plots/gully_Class0_Teff_logg_results.pdf}
 \caption{Pre-main sequence and protostar evolution comparison between theory (left panel) and obseration (right panel).  The 0.02$-$1.2 $M_{\odot}$ evolutionary model tracks are from \citet{baraffe15}, spanning 1 Myr to 100 Myr isochrones; protostars are expected to sit below the 1 Myr isochrone in the region of parameter space demarcated with a yellow dashed ellipse.  The observations show coarse agreement with the models-- measurements of the Pleiades from \citet[green KDE]{cottaar14} cluster around the 100 Myr isochrone, although extend into higher-than-predicted $\log{g}$ for hotter stars.  The younger IC348 sources \citep[purple KDE]{cottaar14} cluster with a large spread above and below the $\sim10$ Myr isochrone.  The source S 68N has a broad posterior PDF (Experiment 4-teal KDE, Experiment 3- red/brown KDE) placing its maximum a-posteriori estimate inside the range of protostars, regardless of experimental assumptions.  The black dots are Class I protostars from \citet{2005AJ....130.1145D}, showing a large range in measured properties with relatively large uncertainties in $\log{g}$.}  
 \label{fig:theory_v_obs}
\end{figure*}



\section{Discussion}


\clearpage
\pagebreak


\appendix

\section{Adaptations of Starfish to handle absolute flux and solid angle}

\acknowledgements
We thank the Keck telescope staff. 

{\it Facilities:} \facility{Keck (NIRSPEC)}

{\it Software: } 
 \project{pandas} \citep{mckinney10},
 \project{emcee} \citep{foreman13},
 \project{matplotlib} \citep{hunter07},
 \project{numpy} \citep{vanderwalt11},
 \project{scipy} \citep{jones01},
 \project{ipython} \citep{perez07},
 \project{starfish} \citep{czekala15},
 \project{seaborn} \citep{waskom14}

\clearpage

\bibliographystyle{apj}
\bibliography{ms}

\end{document}
